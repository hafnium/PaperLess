\section{System Overview}
\label{sec:overview}

In this section we present the system-level implementation of our
approach. We describe the various components to our system and
describe the interactions between them. This is not meant to be a
formal specification, but rather a more informal description of our
specific implementation. For a more formal description of our system
please see the supplemental material.

We now describe our system in two parts. The first is the receipt
processing group, which is responsible for receiving, processing, and
saving receipts from customers. The second part of this section
describes the backend data warehousing group. We use this group to
store the receipt data from the processing group for future access and
analysis.

\subsection{Receipt Processing}
\label{sec:overview.processing}

An crucial property of our entire system is that it scale well with
the demands of the customers. As such, our design of the receipt
processing pool reflects this goal. The vision of our system is that
it will able to efficiently and reliably serve many millions of
requests with little or no degradation of service. We set out to
accomplish this by attempting to minimize possible downtime and/or
degradation of service. Our approach is twofold. First, we have
designed our system to be redundant. If any particular subsystem is
faulty for any reason, then our entire system is able to adapt and use
other resources to ``fill in the gaps.'' Unfortunately redundancy is
only half the battle. Even a highly redundant system can fail to
satisfy the needs of a growing number of requests. Therefore, as our
second goal, we wish to create a system that is also highly scalable.

Figure~\ref{fig:overview} is an overview of our system. The entry
point to the entire processing portion is the gateway. All receipt
requests arrive at the gateway. From there, our ProcessDistributor
attempts to schedule the receipt to be processed by a particular
ReceiptProcessingServer. Each receipt is assigned its own thread that
will be responsible for communicating with a particular processing
server. The threads themselves are designed to be lightweight and
completely independent. This allows the gateway to handle a large
number of receipts at once. Although this puts a critical confidence
in the gateway servers, we are able to provide redundancy and
scalability by using a simple scheme that uses DNS to choose an
alternative gateway if one goes down or to simply add a new one to the
pool and add a DNS entry if demand is higher.

One important issue for the gateway is its ability to
load-balance. This is accomplished by keeping a list of recently
scheduled jobs for each server. If a server does not respond to a
request, it is put on a blacklist and another server is used
instead. Periodically the distributor will check its blacklist to see
if any of those servers have come back online. New servers can be
easily added and removed by simply updating these lists allowing us to
easily scale our entire system as needed.

The servers themselves can be run on effectively any hardware with an
Internet connection. The server process simply polls its listening
socket for any connections from the gateway, and upon receiving a
request, processes that request and returns the result. Specifically,
when a processing server receives a request to process a receipt, it
first loads the receipt from a redundant distributed file system,
performs optical character recognition (OCR), parses the result into a
set of lines, and saves the information to our data warehouse.

\subsection{Database Overview}
\label{sec:overview.db}

\remark{Describe the server architecture (RPC, Sockets, etc). Also
desribe the distributed nature of the whole system. MySQL
server. Gateway. Redundancy. Basically tie in the information from the
requirements section. This section should read like the specification
but not be as formal. Make sure to ref the spec document though. Maybe
take some time to discuss out Rapid Collaborative Refinement
procedure.}