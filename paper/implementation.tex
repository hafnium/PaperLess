\section{Implementation}
\label{sec:implementation}

In the spirit of good software engineering practices we made every
effort to abide by a rigorous approach during development. The outcome
of our efforts was a similar, but slightly tailored version, of the
agile method. We dub our method rapid collaborative refinement. In
essence, due to our small development team (2), close proximity, and
unqiue abilities we were able to very quickly refine a particular
piece of code or refine our approach by alternating between personal
interaction and separated investigation. The lifecycle for a
particular system-level decision was therefore reduced to a small
overhead leaving us more time to focus on developing useful software
and well-constructed software. This approach is depicted graphically
in Figure~\ref{fig:RCR}. Notice that we are able to more tightly close
the loop on developing a particular aspect of the system due to our
methodology.

\input{RCR-fig}

Our development environment consisted of several testing machines
running Linux and Mac OS X, the Eclipse IDE, git source code
management through github.com, mySQL Cluster Server, and MyEclipse for
making logical diagrams. All code was written in JAVA so that we
would have more flexibility in deplyoment options.

We know describe in detail the actual software behind our system. In
Section~\ref{sec:overview} we discussed the high-level overview of our
system. Now we will explain exactly what we did, and what we didn't
do. We will start by describing our implementation of the processing
group and then move on to the data warehousing group.

\subsection{Processing Group}
\label{sec:implementation.processing}

From a high-level perspective, the processing group is responsible for
processing a set of receipts in an efficient and scalable manner. For
this reason we chose to separate the role of process assignment and
process fulfillment. Our ProcessDistributor acts as the assigner. It
polls its listening socket and waits for a request to process a
receipt. When it receives a request, it simply puts the request into a
FIFO queue. A separate thread continuously checks the queue for
elements and upon finding one creates a new handling thread that will
process the item.

Each handling thread executes a remote procedure call (RPC) to a
ReceiptProcessingServer. The particular server chosen is determined by
the ProcessDistributor and is currently just a round-robin
selection. The RPC itself was written from scratch so that we would
have the flexibility to create a lightweight object tailored to our
very simple needs. RPC's are implemented as serializable objects that
can be transferred via a socket to a ReceiptProcessingServer. Upon
receiving the RPC, the processing server extracts the necessary
information from the parameter field, saves the relevant data to the
mySQL backend (see Section~\ref{sec:implementation.db}), and then
returns a response.

As discussed in Section~\ref{sec:overview.processing} our goal was to
be able to process imags of receipts. These images would be stored on
a distributed file system and the RPC parameters would just consist of
a file handle to the particular receipt image to be processed. Due to
complications, we were unable to accurately perform OCR on
images. Section~\ref{sec:results} describese in more detail exactly
what these complications were. Nevertheless, we still wanted to
implement a working system if not complete. In our current
implementation, the parameter to the RPC is a string representation of
a receipt. Our system then parses the string representation (just as
it would parse the output of OCR) for the item name, price, quantity,
seller, user information, etc. This information is saved to our mySQL
cluster database through the JAVA mySQL connector.

If at any time there is an error on the processing server, a message
is sent bck to the distributor notifying it of the error. The
distributor is then able, depending on the error, to re-queue the
receipt for processing again. Some situations where this would be
appropriate would be if the processing server happens to lose
connectivity with the mySQL database.

The beauty of the system is that independent functional parts are
separated by logical and information boundaries. The distributor knows
only a list of possible processing servers. It holds the queue and
iteratively attempts to empty the queue by sending RPC's through its
handling threads to a processing server. The processing servers, in
turn, know nothing about the distributor. They simply listen for data
on their sockets, process requests that come to them, and return the
results. Similarly the database has no knowledge of the specifics of
neither the processing servers nor the distributor. This setup allows
us to very easily scale the entire operation as needed. We need only
notify the distributor that another potential processing server is
available. All other parts are completely agnostic to its pressence or
lack thereof.

\subsection{Data Warehousing Group}
\label{sec:implementation.db}

The data warehousing group is comprised of a distributed mySQL
Cluster.  The Cluster consists of multiple management nodes which
oversee a variable number of storage nodes running on commodity
systems.  The level of redundancy is maintained by the management
nodes and can be adjusted as necessary.  Data is copied to multiple
storage nodes to reflect the desired level of redundancy to ensure
data integrity.

As a proof of concept, we configured a small three system cluster with
one system acting as both a management node and storage node.  This
allowed us to run experiments using different levels of redundancy and
simulated failures.  In the early stages of the project we simulated a
distributed cluster using a single machine.  While this provided
evidence that the system would function as expected when deployed, we
were determined to get a truly distributed configuration running.
Although our cluster is small, we found it sufficient to run large
workloads for experimental purposes.  It was far more impressive to
see in action than the simulated version.

To improve access time and remove redundancy from the database we
spent a good portion of our planning phase devising an appropriate
schema for the database.  Our schema is organized as tree structure
containing six distinct tables: Line, Receipt, User, Location, Store,
City.  We will now discuss these in further detail.

The Line object contains all information pertaining to a single item
from a receipt.  This includes a the item name, the price per unit,
and the quantity purchased.  The Line object also contains a reference
to a Receipt object.  The Receipt object consiste the date of the
purchase, and references to both a User object and a Location object.
The User object simply contains the registration information for a
particular user.  It is used to determine which receipts belong to
which users.  The Location object contain refrences to both a Store
object and a City object.  These references are used to accomodate
retail chains that may have more than one location in a single city.
The City object contains fields for city, state, and zip.  This
reduces redundancy when the system contains multiple clients in the
same city.  The Store object contains the name and address of the
retailer.  This is also to reduce redundancy in the database for
retail chains.  Additionally, if a company were to change its name,
this would only require updating a single record.

