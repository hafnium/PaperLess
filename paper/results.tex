\section{Results}
\label{sec:results}

Our goal was to create a system that could process receipts in a
scalable and reliable way. We have defined scalability as the ability
to quickly and seemlessly add or remove processing
entities. Reliability refers to the persistance of data after it has
been received by our system. We have described our system and its
specific implementation to date in Section~\ref{sec:overview} and
Section~\ref{sec:implementation}. Now we set out to evaluate our work.

First we will consider scalability. In order to test the scalability
of our system we required a large test set. As mentioned in
Section~\ref{sec:implementation.processing}, we were unable to get
adequate results from an optical character recognition package we
tried. We tried Tesseract and SimpleOCR on a small set of test
receipts. Although according to 1995 UNLV Accuracy Test, Tesseract is
one of the best OCR packages available. However, our initial tests
deemed it unable to extract text correctly from receipts most likely
due to the poor resolution available and the non-standard layouts
present in receipts from many different companies. Therefore, we chose to consider receipts as strings with fields corresponding to what would have been extracted from an OCR package. A typical receipt line then takes the form:

\begin{centering}
\{receipt:RECEIPT ID\} \{item: ITEM NAME\} \{price: ITEM PRICE\} \{quantity: QUANTITY\} \{store: STORE ID\} \{date: PURCHASE DATE\}
\end{centering}

For our tests, we generated 100,000 lines of this format. We then
wrote a simple test harness that looped over the list and sent each
line to the process distributor gateway. We ran this test with only
two processing servers available. Unfortunately, the bottleneck was
the mySQL cluster. Since each connection to a mySQL database requires
opening a file, the number of allowable open files was quickly
saturated. However, as we scaled back the number oftest receipts to
5,000, we saw that we could efficiently process receipts with our
system. In addition, if we removed one of the servers by killing it,
the process distributor properly sent the receipts to the other
available servers. Likewise, if we added a new server, the disitrbutor
was able to adapt and send receipts to the newly added one too.
