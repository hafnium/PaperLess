\section{Introduction}
For this project, we proposed to create a system for providing digital receipts for everday transactions in place of paper receipts which we consider to be an unnecessary waste.  In the initial phase of this project we enumerated a small set of general goals for the system.  We wanted the system to be invisible to both the customer and the cashier to minimize both training and confussion during the transition to the Paperless system.  Due to the large number of transactions that occur daily, we knew from the start that the system had to be scallable and would likely need to be distributed.  Additionally, we imagined that companies utilizing the system would want to aggregate statistical data on customer activity.  This would require permanent storage of the receipts in our digital warehouse.\\
During the next phase of the project we examined the potential risks associated with our undertaking and determined what could reasonably be accomplished in a semester.  Since we did not have direct access to a functional point or sales (POS) system nor the means to procure one we decided to focus primarily on the server-side implementation of the system.  All client-side work has been based on personal knowledge from years of working with the Aloha POS system and remains as future work for the project.\\
In this paper, we discuss the project as it was proposed and the progress that we have made thus far.  In section 2 of the paper we present an overview of the important requirements for the system and discuss their significance.  Section 3 provides an overview of the system from a structural perspective.  This section is not intended to detail the inner-workings of the system but rather the components that comprise it.  In section 4, we discuss our implementation methodology as it pertains to Software Engineering.  While our goal was to create a functional product, we were primarily concerned that whatever progress we made over the course of the semester was accomplished through the use of good Software Engineering practices.  Section 5 discusses how we tested the system once it was functional and what we learned from our initial testing that will be addressed in future work.  In section 6, the issues that we ran into during development are revealed.  While we tried to minimize risk from the beginning of the project, we were prepared to deal with unexpected problems.  Section 7 addresses our future work on the system and we conclude the paper in section 8 with our reflections on the overall experience.
