\section{Introduction}
As the world becomes more environmentally conscious, it becomes
increasingly necessary to revise the financial transaction
infrastructure to accommodate the trend.  Although paper receipts
create very little waste on a per transaction basis, the volume of
receipts being printed daily has a negative effect on the environment.
The Paperless system aims to reduce this wastefullness by providing
customers with the option of receiving a digital receipt in place of
its paper counterpart.  Customers can then access their digital
receipts through the Paperless system via an online web interface,
providing an environmentally friendly alternative to paper receipts
with the added benefeit of a scallable distributed backend capable of
compiling statistical data on customer purchasing activity.

In the initial phase of this project we enumerated a small set of
general goals for the system.  We wanted the system to be invisible to
both the consumer and the seller to minimize both training and
confussion during the transition to the Paperless system.  In
addition, due to the large number of transactions that occur daily, we
knew from the start that the system had to be scallable and would
likely need to be distributed.  Additionally, we imagined that
companies utilizing the system would want to aggregate statistical
data on customer activity.  This would require permanent storage of
the receipts in our digital warehouse.

During the next phase of the project we examined the potential risks
associated with our undertaking and determined what could reasonably
be accomplished in a semester.  Since we did not have direct access to
a functional point of sale (POS) system nor the means to procure one
we decided to focus primarily on the processing and warehousing
aspects of the system. 

In this paper, we discuss the project as it was proposed and the
progress that we have made thus far.  In
Section~\ref{sec:requirements} of the paper we present an overview of
the important requirements for the system and discuss their
significance.  Section~\ref{sec:overview} provides an overview of the
system from a structural perspective.  This section is not intended to
detail the inner-workings of the system but rather the components that
comprise it.  In Section~\ref{sec:implementation}, we discuss our
implementation methodology as it pertains to Software Engineering.
While our goal was to create a functional product, we were primarily
concerned that whatever progress we made over the course of the
semester was accomplished through the use of good Software Engineering
practices. Section~\ref{sec:results} discusses how we tested the
system once it was functional and what we learned from our initial
testing that will be addressed in future work.  In
Section~\ref{discussion}, the issues that we ran into during
development are revealed.  While we tried to minimize risk from the
beginning of the project, we were prepared to deal with unexpected
problems.  Section~\ref{sec:future} addresses our future work on the
system and we conclude the paper in Section~\ref{sec:conclusion} with
our reflections on the overall experience.
