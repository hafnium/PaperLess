\section{Conclusions}
\label{sec:conclusion}
We set out to develop a distributed and scalable system for providing storage and access to digital receipts.  We feel that we have accomplished our goals for this phase of the project.  The use of digital receipts has many benefits.  It reduces paper waste, reduces paper cost, receipts are stored permanently, and statistical data can aggregated.  The system that we developed provides these benefits.

While many of our goals for this project were focused on the actual system, it was essential to the success of this project that our development was guided by good Software Engineering practices.  Using rapid collaborative refinement, we were able to quickly develop requirements for the system and iteratively develop new working components while improving existing ones.  This ensured that at any point in our development we always had a working protype.  We preferred this method over throw-away prototyping because we didn't want to waste what little time we had on non-system code.  Additionally, our method eliminated development bottle-necks because the coders were always working on code separated by a well defined interface.

To conclude this paper we would like to discuss what we learned over the course of this project.  Identifying risks early in the development process helps to determine what is realistically possible in a given time frame.  If we had attempted to develop the client-side and server-side systems in parallel we would have likely ended up with an elaborate communications system that did minimal receipt processing if any.  We also confirmed that rapid collaborative refinement is a viable method for Software Development as long as communication is emphasized.  At all stages of the development process there was a clear understanding of what each developer was working on.  This eliminated the frustration of attempting to merge changes in the git repository and reduced the chance of breaking the build.


