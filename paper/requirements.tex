\section{Requirements}
\label{sec:requirements}
In accordance with good Software Engineering practices, we compiled a set of high-level system requirements during the planning phase of the project to guide development over the course of the semester.  As the project progressed, the set of requirements was updated to reflect unforseen changes that occured during the development process.  In this section, we discuss the requirements that we consider to be the most significant for the Paperless system.  We divided our requirements in five distinct groups: User requirements, Reporting requirements, System and Integration requirements, Security Requirements, and User Interface requirements.

\subsection{User requirements}
\label{sec:requirements.user}

From and end-user perspective, the most crucial aspects of the system is that it's easy to use and always available.  To accomodate these requirements, we elected to use a simple web interface to provide users access to their digital receipts.  We consider it low-risk to assume that the average person is familiar enough with the internet to be able to navigate a simple website with little effort.  Additionally, we require that the system be distributed with no single points of failure and that data be stored redundantly.  To ensure no single point of failure for receipt processing, we developed a distributed receipt processor that load balances over available processing nodes.  These processing nodes can be started and shutdown on the fly and ensures that receipts can be processed as long as one node is running.  To ensure no single point of failure for the receipt warehouse, we elected to use a mySQL Cluster to store the receipts.  As with the receipt processor, storage nodes can be added and removed from the cluster on the fly and the database will continue to function.  Additionally, we can adjust the level of redundancy over the cluster as necessary.  Redundancy is managed by the mySQL Cluster natively.  If a storage node is removed from the cluster, the data that was lost is automatically copied to a different running node to restore redundancy.

\subsection{Reporting requirements}
\label{sec:requirements.reporting}

From the onset of this project, we wanted the system to have the ability to aggregate statistical data on customer activity.  This data is not only important for us to analyze system behavior, but is of great benefeit to retailers that wish to better accomodate their customers or develop targeted advertising campaigns.  This requirement was essentially fulfilled through the use of mySQL Cluster as the backend of our storage warehouse.  Using the native operations provided by mySQL Cluster, simple scripts can easily extract data from our storage warehouse that can then be processed offline.  Each receipt is broken down into its atomic elements prior to database insertion which eliminates the need to pre-process data before it can be analyzed.  Additionally, receipts are timestamped at the time of the transaction as well as when they are inserted into the database.  This allows us to analyze system performance and locate bottlenecks in the processing and storage system.

\subsection{System and Integration requirements}
\label{sec:requirements.system}

The system and integration requirements for the system are primarily to ensure the integrity of the Paperless system.  The biggest risk while the system is running is the loss of data during transmission.  This can occur at various stages of the storage process.  The receipt must first be sent from the POS system to our processing servers.  Then it must be sent from the processing servers to the storage servers.  If the receipt is lost during either of these stages, the integrity of the system is compromised.  To mitigate this risk, we ensure that all transmissions are confirmed on the receiving end before they are considered successful.  If a problem arises during transmission, the transaction is ignored and placed back in the queue to be re-transmitted.

\subsection{Security requirements}
\label{sec:requirements.security}

From a security perspective, the secrecy of our users personal information is of utmost importance.  All passwords are stored as MD5 hashes in the database, unnecessary ports are blocked by a firewall, and while not implemented at this stage, all data will be encrypted prior to transmission.  Good coding practices will be used to develop the web interface to prevent mySQL injection attacks, buffer overflow attacks, DOS attacks, and other known vulnerabilities.

\subsection{User Interface requirements}
\label{sec:requirements.interface}

One potential risk for the web interface is slow response time.  We envision a massive ammount of data being stored in our database which makes targeted receipt retrieval an expensive task.  For this reason, we will require that the mySQL Cluster be sufficiently distributed to balance requests and mitigate the bottleneck from disk access.  Additionally, the management nodes in the mySQL cluster cannot be used as storage nodes.  They should only be concerned with re-directing requests to the appropriate storage nodes.
